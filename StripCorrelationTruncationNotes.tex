\documentclass[%
    reprint,
    amsmath,amssymb,
    aps,
    floatfix,
]{revtex4-2}

\usepackage{braket, nccmath}
\usepackage{graphicx, verbatim}
\usepackage{dcolumn} % Align table columns on decimal point
\usepackage{bm} % Bold math
\usepackage{hyperref} % Add hypertext capabilities
\usepackage{dsfont} % Add double stroke fonts
\hypersetup{
    colorlinks=true,
    linkcolor=blue,
    citecolor=blue,
    urlcolor=blue,
}

\usepackage{amsthm}
\theoremstyle{plain}

\DeclareMathOperator{\Tr}{Tr}
\newtheorem{theorem}{Theorem}[section]
\newtheorem{lemma}{Lemma}[section]
\newtheorem{definition}{Definition}[section]
\pdfstringdefDisableCommands{\def\eqref#1{(\ref{#1})}}

\begin{document}

\title{2-D Metallic State Strip Correlation Truncation Notes}

\author{Hing Nin Chan}
\affiliation{Department of Physics, Hong Kong University of Science and Technology, Hong Kong, China}

\begin{abstract}
    Since the 2-D metallic state has a entanglement entropy proportional to $L log L$, where $L$ is the system
    size. Hence it is expected that the renormalization group have a branching tree structure. In this note, we 
    will describe how to obtain the correlation matrix for the 1-D metallic states from the 2-D metallic system.
\end{abstract}

\maketitle

\section{Introduction}
The 2-D metallic state can be fully described by its correlation matrix denoted as $C_{2D}$, in our context it can 
be described as a Fock Map, which is a rank-2 tensor that transform one Hilbert space to another. Thus we can denote 
$C_{2D}: \mathcal{R} \rightarrow \mathcal{R}$, where $\mathcal{R}$ is the Hilbert space of the 2-D metallic state in 
the physical space. A similar description in the reciprocal space can be used $C_{2D}: \mathcal{K} \rightarrow \mathcal{K}$ 
where $\mathcal{K}$ is the Hilbert space of the 2-D metallic state in the reciprocal space.

\section{Strip Correlation Matrix}
Since the 2-D metallic state we are considering have a Fermi surface perpendicular to the reciprocal space vector 
$\vec{v}_{g} = [1,1]^T$ in the square space, thus we expect the 1-D metallic state to be localized within strips parallel 
to $\vec{v}_{g}$. Which we can express the 1-D metallic state in terms of the basis that is translational invariance along 
the $\vec{v}_{g}$ direction.

\newcommand\blockedcorrelations{C_{2D}^{\mathcal{B}}}

To have enough degrees of freedom to perform our strip metal distillation process, we will perform a uniform scaling with a factor 
of $4$. We will denote the scaled correlation matrix as $\blockedcorrelations$.

In order to obtain the correlation matrix for a single 1-D metallic state parallel to $\vec{v}_{g}$, we have to first find 
the correlation matrix $\blockedcorrelations$ expressed in terms of the translational invariance of the $\vec{v}_{g}$ direction. In order 
to complete then span of the 2-D space we use a non-trivial method with Smith normal form to find another real-space basis 
to complete the 2-D space.

\begin{equation}
    \label{eq:StripRealSpaceBasis}
    \mathcal{R}_{S} =
    \begin{bmatrix}
        1 & 0 \\
        1 & L_y
    \end{bmatrix}
\end{equation}

$\mathcal{R}_{S}$ is the column basis and $L_y$ is the number of lattice sites in the $y$ direction, actually one can also 
use the $x$ direction since it will still span the same 2-D space.

\newcommand\scaledcorrelations{C_{2D}^{S}}

The new crystal represented in $\mathcal{R}_{strip}$ basis will have the size of $1 \times L_x$ since all sites in the $y$ direction 
are grouped into a single unit cell. Denote the transformed correlation matrix as $\scaledcorrelations$.

After that we will perform a strip restriction on $\scaledcorrelations$ with a real-space strip that have a width correlated with the 
correlation length of the 1-D metallic state in the perpendicular direction of $\vec{v}_{g}$, and we can find a strip unit-cell with 
length of $norm(\vec{v}_{g})$. This process can be done by applying a Fourier isometry on $\scaledcorrelations$ back to the real space.

\newcommand\restrictedfourier{\mathcal{F}_{\mathcal{R}}}

\begin{equation}
    \label{eq:StripUnitCellRestrictedCorrelations}
    C_{\mathcal{R}} = \restrictedfourier^{\dagger} \scaledcorrelations \restrictedfourier
\end{equation}

\section{Reasons to perform Strip Correlation Matrix Truncation}

After that we will perform a Fourier transform with translational invariance along the $\vec{v}_{g}$ direction to obtain the strip correlation 
matrix of the single 1-D strip region, denote it as $C_{strip}$. This is not the correlation matrix resembling the 1-D metallic state yet, since 
by inspecting the correlation spectrum of this matrix we can see that it is not a pure state, thus we will have to locate the degrees of freedom 
that attributes a pure state within this spectrum. Technically we will do it by applying a threshold filtering to the correlation spectrum such 
that only the degrees of freedom that have correlation values close to $0$ or $1$ will be selected. Then we will obtain the correlation matrix of 
the 1-D metallic state $C_{metal} = \mathds{1} - \mathcal{P}_{pure}$ where $\mathcal{P}_{pure}$ is the projector from the selected modes.

When one observe the correlation spectrum of $C_{metal}$, we will see a pole at the $\Gamma$ point in the reciprocal space of the 1-D strip space, 
in such we cannot identify a set of filled degrees of freedom that can be used to describe the 1-D metallic state. From the properties of Fourier 
transform, we know that a pole in the reciprocal space means there are delocalization behavior in the real space, thus we can perform a truncation 
of the correlation matrix in the real space, which is equivalent to cutting all correlations that have range longer than some real space distance, 
and that distance is determined by the correlation length of the 1-D metallic state in the direction of $\vec{v}_{g}$.

\section{Methodology}
Going back to $C_{strip}$, we can construct a free-fermion Hamiltonian $H_{frozen}$ that have all the frozen degrees of freedom which are the ones 
with correlation value close to $0$ or $1$.

\begin{equation}
    \label{eq:StripGlobalDistillHamiltonian}
    H_{frozen} = \sum_{k} - |\psi_{k}^{frozen} \rangle \langle \psi_{k}^{frozen}|
\end{equation}

From equation \ref{eq:StripGlobalDistillHamiltonian} we can see that all frozen modes will span the non-zero $-ve$ bands and the remaining modes 
which are not localized within the strip region, will span the null space.

Then we can perform the truncation by applying a Fourier isometry on $H_{frozen}$ to restrict the Hamiltonian to a region with radius 
of the truncation length. This region normally will be a enlongated unit cell of the strip crystal, when we perform the Fourier transform with 
translational invariance along the $\vec{v}_{g}$ direction, the resulting Hamiltonian $H_{frozen}'$ will have multi-counting since the region overlaps 
under translation with $\vec{v}_{g}$, thus we have to manually remove the multi-counting.

By inspecting the spectrum of $H_{frozen}'$ in the reciprocal space of the strip crystal, we can see that there will be a set of bands in the lower 
eigenvalues, those will be the frozen degrees of freedom corresponding to the 1-D metallic state, and thus we can construct a pure state correlation matrix 
by $C_{metal}' = \mathds{1} - \mathcal{P}_{frozen}'$, which will be the correlation matrix of filling all the 1-D metallic single fermion states.

\end{document}
